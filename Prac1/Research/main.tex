\documentclass{article}
\usepackage{enumitem}
\usepackage{hyperref} % Add hyperref package for URL support

\title{Research Questions}
\author{Quintin D'hotman de Villiers (u21513768)}
\date{March 2024}

\begin{document}

\maketitle

\section{Introduction}
\subsection*{COS 333 Prac1}

\begin{enumerate}
  \item Turing completeness refers to a system that can compute or solve any computational program.\cite{what-is-turing-complete} In the context of \LaTeX, Turing completeness refers to \LaTeX's ability to be programmed to compute almost any arbitrary computational problem.\cite{latex-turing-completeness} A possible advantage is that the system is versatile as it can be used to compute/solve any computational problem. A possible disadvantage of this is unpredictable behaviour.\cite{tex-turing-complete} With infinite calculations or possibilities, it may be difficult to account for all possible outcomes (Rice's Theorem\cite{rice-theorem}).
  
  \item An Esolang is a programming language that is meant to test the boundaries of coding languages either through actively being difficult to code in or made as a joke or to show off interesting/strange ideas.\cite{esolang-wiki}

  \item Against Esolang:
    \begin{enumerate}
      \item These languages are oftentimes made as a joke as they tend to satirise or parody cultural references, memes and other strange references or jokes.
      \item These languages oftentimes tend to be made to address problems that are of no practical importance or address issues that need not exist.
      \item They can be very difficult to use, the buy-in to use these programs either is too daunting due to the complexity of them, or not timely in the use of them as it may sometimes take a long time to learn/use and are not widely supported.
    \end{enumerate}
\end{enumerate}

\begin{quote}
For Esolang:
\end{quote}

\begin{enumerate}
  \item[a.] These languages, although amusing and very strange, can be used to illustrate the bounds of computation and what code is capable of such as through Turing Tarpits, whereby a language is made to be Turing complete using as few instructions as possible.\cite{turing-tarpit}
  \item[b.] Esolangs may be used to test the creator's ability to create a language that can solve or compute complex or arbitrary issues, ranging from computing Hello World to quantum computing, to languages that are incapable of being processed.\cite{hello-world-esolang}\cite{category-quantum-computing}
  \item[c.] These languages allow non-standard thinking and approaches to solving problems in programming as well as working within a set boundary or lack thereof in solving these problems.
\end{enumerate}

\begin{enumerate}
  \item[4.] \textbf{Colonoscopy}: This is a humorous esolang created by Dom Davis in 2016. It is based on Brainf*ck except the syntax consists of braces and semi-colons, and each instruction is terminated with a semicolon. The instructions in Colonoscopy consist of moving a pointer left or right, incrementing or decrementing memory cells under the pointer as well as inputting or outputting values stored in the cells at the pointer. This language is Turing complete as it is a variant of Brainf*ck, which is known to be Turing-complete.\cite{colonoscopy-esolang}
  
  \textbf{i like frog}: I like frog is a joke esolang designed in 2020 by user `Apollyon094' based on a Reddit post depicting a keyboard with only 3 keys, labelled `i, `like' and `frog' respectively. The language is a set of 16 commands, making up `i', `like', `frog' and the newline character with line breaks denoting the end of a command. The language uses a tape-based memory system with commands to move across cells as well as to insert into or output values from those cells. I like frog is Turing complete as it can create infinite loops and can be compiled to Brainf*ck variants.\cite{i-like-frog-esolang}
  
  \item[5.] A programming language is considered a language that is used to write programs, scripts and instructions for execution by the computer.\cite{programming-language} Bash can run scripts, and users can write instructions in Bash for execution by the computer. A reason why it may not be considered a programming language is that it is a shell, allowing users to interface with the computer, executing scripts essentially acting as part of the computer in the above definition.\cite{what-is-shell}
  
  \item[6.] Functional and logic paradigms.\cite{alf-paradigm}
  
  \item[7.] Visual Logic makes use of blocks and flow arrows to visually represent the flow and logic of a program. It allows the user to visually determine how the blocks behave, such as input/output, calculations, if statements, looping and visually simulate a program such as that of C or Java.\cite{visual-logic} The advantage of this is that it is able to teach the logic of programming and concepts without needing to know the syntax of a traditional programming language, the disadvantage is that users may not understand the syntax used in programming to create this behaviour.
  
  \item[8.] It helps to identify memory errors in a system, such as access to memory that is uninitialized or unaddressable.\cite{drmemory}
\end{enumerate}

\bibliographystyle{plain}
\bibliography{references}

\end{document}
